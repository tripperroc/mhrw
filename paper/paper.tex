\documentclass{article}

\begin{document}
\emph{Feature extraction} is the conversion of raw data into the feature space of a machine learning model. 
%Often the performance of machine learning algorithms, in terms of both accuracy and resource consumption benefits from 
An often necessary---and, in terms of both better accuracy and resource consumption, subtask is to reduce the dimensions
(or information) of the raw data. For example, support vector machines often perform better on natural language data when stop
words or removed, or when the vocabulary space has been reduced by, say, principle components or latent Dirichlet analysis.

In this paper we explore the use of graphical constraints to reduce the dimensionality of feature space of social media data.


In social media data, a crucial decision is how to extract a social network from the data. One can then apply any number of 
machine learning algorithms that are well suited to network data, such as label propagation or Baye's networks (often by using a small 
number of network templates we global contraints between them). Traditionally, machine learning has focused on how learning
outcomes differ as the learning model varies and the process of converting raw data into the learning model feature space is taken 
for granted. In this paper, we show that with graphical data, feature extraction based on global network features
such as core number, Cheeger constant, can lead to dramatic improvement in learning situations, even when the actual learning algorithm and 
and feature space remains the same. 
\end{document}
